%% LyX 2.1.0 created this file.  For more info, see http://www.lyx.org/.
%% Do not edit unless you really know what you are doing.
\documentclass[12pt,letterpaper,english]{article}
\usepackage[T1]{fontenc}
\usepackage[latin9]{inputenc}
\usepackage{color}
\usepackage{babel}
\usepackage{url}
\usepackage{amssymb}
\usepackage{cancel}
\usepackage{setspace}
\doublespacing
\usepackage[unicode=true,
 bookmarks=true,bookmarksnumbered=false,bookmarksopen=false,
 breaklinks=false,pdfborder={0 0 0},backref=false,colorlinks=true]
 {hyperref}
\hypersetup{pdftitle={Chemlogic: A Logic Programming Computer Chemistry System},
 pdfauthor={Nicholas Paun}}
\usepackage[dot]{bibtopic}

\makeatletter

%%%%%%%%%%%%%%%%%%%%%%%%%%%%%% LyX specific LaTeX commands.
\pdfpageheight\paperheight
\pdfpagewidth\paperwidth

\newcommand{\noun}[1]{\textsc{#1}}

%%%%%%%%%%%%%%%%%%%%%%%%%%%%%% User specified LaTeX commands.
\usepackage{units}

\usepackage{fancyhdr}
\pagestyle{fancy}
\usepackage{lastpage}
\lhead{}
\rhead{}
\renewcommand{\headrulewidth}{0pt}
\cfoot{\fontsize{8pt}{8pt} \selectfont 2014-04-30 --- Nicholas Paun: Chemlogic --- \thepage\ of \pageref{END}}

%\usepackage{anysize}
%\marginsize{1in}{1in}{0.5in}{0.5in}

\usepackage[margin=1.0in,includefoot]{geometry}
\topmargin 0pt
\headsep 0pt
\headheight 0pt
\usepackage{titlesec}
\titleformat{\section}
  {\normalfont\fontsize{13.5}{15}\bfseries\scshape}{\thesection}{1em}{}
\usepackage{url}

\makeatother

\begin{document}

\title{\vspace{-0.75in}
Chemlogic:\\
 A Logic Programming Computer Chemistry System}


\author{Nicholas Paun}


\date{~}

\maketitle
\input{/home/np/.lyx/templates/master.tex}

 \thispagestyle{fancy}

\vspace{-1in}



\section*{$*$ Abstract}

\noun{Chemlogic is a logic program} for balancing chemical equations
and converting chemical formulas to and from chemical names, using
a database of chemical element and polyatomic group information, a
set of grammars, and a linear equation solver. Chemlogic can detect
and provide guidance for resolving syntax and other errors and has
a user-friendly Web interface.\cite{wielemaker:tplp2008}


\section*{Background}

\noun{In high school Chemistry}, students learn to write formulas
and names for chemical compounds and to write and balance chemical
equations. These concepts are simple, but implementing a program to
do this was an interesting task. Algorithms were researched, adapted
to chemistry problems and implemented in Prolog to create a program
that could be useful in education.


\section*{Design and Implementation}

\noun{User input is parsed} into a form that can be easily manipulated
and transformed. A parser recognizes a formal grammar describing valid
user input. In Prolog, parsers are implemented using DCGs (Definite
Clause Grammars), which provide a simplified syntax for creating logical
clauses that process a grammar using difference lists, an efficient
representation (e.g. concatenation in $\mathrm{O}(1)$).\cite{triska:dcg}

Difference lists consist of an instantiated part (the head) and an
uninstantiated part (the tail), which is always paired with the rest
of the lists. Difference lists are a complex concept and are usually
abstracted away by DCG syntax. 

In DCG clauses, the head of a grammatical rule is unified with the
input and any remaining data is unified with the tail, which is passed
to the next clause until the parsing is completed. If any clause fails,
Prolog will backtrack to find another clause that can satisfy the
grammar. If no clause is found, then the parsing fails.

To make a useful program it is not enough to simply decide whether
or not a given input conforms to a grammar --- internal representations
must be created. Commonly, Abstract Syntax Trees are used for this
purpose. Chemlogic uses a pseudo-AST to record the structure of an
equation, as well as lists containing useful information (e.g. the
elements contained in an equation). DCGs provide extra arguments for
this purpose.\\
\\
\noun{Balancing of chemical equations} is usually done by inspection.\cite{sandner2008bc}
This process cannot easily be used in a program because it is unsystematic:
as coefficients may be corrected again during the balancing, the order
and steps performed may vary from equation to equation.

Chemlogic uses an elegant process, using a system of linear equations.
One linear equation is created for every element in a chemical equation,
with the number of occurrences of the element in each formula representing
a coefficient, multiplied by an unknown (the chemical equation coefficient).\cite{tuckerman:nyu}
To make the system solvable, the first coefficient is set to 1. The
solution is always reduced to lowest whole number terms.

This process can be made even simpler to program by creating a homogeneous
linear system, where the terms representing reactants have positive
sign and terms representing products have negative sign. These equations
are all equal to 0. These systems are commonly solved by converting
them to a matrix and applying Gaussian elimination.\cite{eigenstate:chembalancerjs}

In Chemlogic, a matrix is produced from structures created by the
parser and lookup tables. The matrix is converted into a system of
linear equations, which is then provided to the built-in CLP(q) facility,
which can solve constraints over rational numbers.\cite{holzbaur:clpq}
This process is less efficient, but allows for code reuse, saving
programmer work.\\
\\
\noun{Syntax errors} cannot simply cause a program to fail --- clear
identification and explanation of an error is necessary. When a predicate
that must succeed for a given input to be valid fails, a syntax error
exception is thrown, containing a code name for the error and whatever
remains in the tail (what could not be parsed). The exception aborts
the execution of the program and is then passed to the error handling
module. It first attempts to localize the error by highlighting only
the problematic part within the tail. Different rules are used depending
on the type of the first character (e.g. an invalid letter suggests
a chemistry mistake, while an invalid symbol suggests a typo). The
combination of an error code and character type is used to find the
correct error message to provide to the user.\\
\\
\noun{Multiple interfaces are supported} in Chemlogic. Currently,
command-line and Web interfaces have been implemented. In order to
show correct symbols and formatting (e.g. subscripts) in each interface,
there is an output formatting module that allows each parser to automatically
use the correct symbols. The error handling module re-throws its syntax
errors to a simple error handler for each interface.\\
\\
\noun{Metaprogramming is an excellent feature in Prolog}. Metaprogramming
allows a program to write or manipulate parts of itself. An important
aspect of metaprogramming is the ability to manipulate code as a data
structure --- this is used to translate simple facts from a database
into various grammatical rules. 

Prolog also allows a programmer to define new operators that extend
the programming language and provide simple syntax for repetitive
tasks. Chemlogic defines an operator that throws a syntax error, if
a predicate fails, and another operator that catches syntax errors
and runs the correct handler for the current interface. Defining operators
makes Chemlogic's code easier to read and understand.

Chemlogic also implements a very simple Domain Specific Language (DSL)
on top of Prolog, using metaprogramming techniques. This DSL is a
proof-of-concept and consists of three rules that provide simple syntax
for a user to query Chemlogic. DSL clauses can be composed into very
simple programs and the full features of Prolog can be combined with
DSL rules, if needed.


\section*{Discussion}

\noun{Performance was analyzed} in Chemlogic by counting inferences
(provided by \texttt{time/1}) used by different algorithms for various
problem sizes. Algorithms were compared on their fixed inferences
(intercept), inferences per item (slope) and to ensure that their
complexities were not exponential.

The time taken by the algorithms used in Chemlogic could not be analyzed
because the difference between the performance of algorithms on typical
problem sizes was immeasurably small.\\
\\
\noun{Further research and development} --- It would be interesting
to compare the balancing algorithm used, with one that attempts to
balance equations by trial-and-error. A brute-force algorithm is logically
unsatisfying and its difficulty increases exponentially ($\mathrm{O}(n^{m})$).
The time taken by repeated trials may be unnoticeable on new computers,
however. 

Currently, Chemlogic attempts to distinguish sub-classes of errors
to make error messages more specific. A topic for future development
would be to substitute pieces of information from incorrect input
into messages, allowing the program to explain exactly what is wrong
with the input, as opposed to a general error message.

The program could be extended to add more chemistry features, including:
structural formulas and the names of more complex organic compounds,
stoichiometric calculations and completion of simple reaction types
(a very complex task).


\section*{Conclusions}

\noun{Prolog was chosen} as the language for Chemlogic because it
has many features that are useful for implementing this type of program. 

Prolog includes support for DCGs, which allow a programmer to implement
a parser using a very simple syntax, without requiring manual parser
writing. Using DCGs, it is easy to write grammatical rules, test them
and add more types of input. Chemlogic used many advanced features
of DCGs and their underlying abstractions.

SWI-Prolog's CLP(q) facility made solving systems of linear equations
extremely simple.

Using a logic programming language, such as Prolog, enables the programmer
to describe the results, instead of the process.\cite{wielemaker:2011:tplp}
In practice, writing in Prolog avoids the need for manual programming
of loops and other imperative constructs, most often resulting in
a well-written, succinct solution requiring few lines of code.

Prolog also has very strong support for metaprogramming, which was
used in a few places in Chemlogic, where the amount of boilerplate
code needed was reduced, making code easier to read.\\
\\
\noun{Chemlogic was successfully implemented} using Prolog, in a well-designed
and modular structure, and could balance chemical equations, convert
names to formulas and vice versa.\label{END}

\appendix
\pagebreak{}

%\backmatter
\pagenumbering{roman}
\cfoot{\fontsize{8pt}{8pt} \selectfont 2014-04-30 --- Nicholas Paun: Chemlogic --- \thepage\ of \pageref{APPEND}} 


\section*{Acknowledgments}

I would like to thank the many people who gave advice and helped with
the project.

I am particularly grateful for the valuable assistance provided by
Dr. Peter Tchir, my Physics, Chemistry and, now, Computer Science
teacher. His help and advice, especially with algorithms and his support
for my Computer Science projects helped make this program possible.

I would also like to thank Mr. Jason Peil for his assistance in designing
and printing the display and Mr. Greg Osadchuk for his input and assistance
regarding the visual presentation.


\section*{Obtaining Chemlogic / Contact}

Nicholas Paun <\url{np@icebergsystems.ca}>

Chemlogic is open-source software. A copy of the program and additional
information is available at \url{http://icebergsys.ca/chemlogic}


\section*{References}

\bibliographystyle{plainurl}
\begin{btSect}{chemlogic}
\btPrintCited
\end{btSect}
\label{APPEND}
\end{document}
